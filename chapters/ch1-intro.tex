\begin{savequote}[8cm]
Le roi est mort, vive le roi!
  \qauthor{\quad--- Traditional French proclamation at the death of one monarch and the ascension of a new}
\end{savequote}

\chapter{Introduction}
\label{ch:1-intro} 

\emph{"The King is dead, long live the King!"}. Thus spoke the Duke of Uzès at the death of every French monarch from 1498 until Louis XVI took the throne (presumably the current duke had other things on his mind during the winter 1792--93). In the original setting, the words express the fundamental premise of hereditary monarchy: at the last breath of the king, the god-given sovereignty of the crown passes to the oldest, living heir; in an altogether different sense, the words fittingly summarise the situation in the field of particle physics. 

In the sub-atomic world, the Standard Model rules supreme in spite of everyone agreeing that it is in fact quite dead: while the successes of the model are numerous, it leaves several phenomena that have observed in the world completely unexplained. Most famously, \emph{gravity} is not included in the model, and no mathematically consistent unification of general relativity and the Standard Model has been found. A related short coming is the lack of an explanation of \emph{dark matter} and \emph{dark energy} within the theory: the Standard Model describes only 5\,\% of the energy content of modern cosmological models of the universe. Of special relevance to this thesis is of course the matter-antimatter asymmetry in our matter-dominated universe, for which there is no explanation in the Standard Model. Furthermore, there are a number of theoretical issues, such as the (relatively) low mass of the Higgs' boson requiring extremely fine tuning of parameter values, and the absence of an explanation for the hierarchy of masses that the Higgs mechanism gives rise to.
When the Standard Model continues to live long in spite of these shortcomings, it is not for want of tries at murder: thousands of physicists at dozens of experiments spend their days looking for physics effects that are \emph{Beyond the Standard Model}, but so far no (statistically significant) experimental results have been obtained that point to a suitable successor, able to resolve the fundamental issues with the existing theory.\footnote{Neutrino masses have been experimentally observed to be different from zero. This can be accounted for by several possible extensions of the Standard Model, which cannot be told apart given current data, but potentially in the near future. However, such extensions are not expected to resolve the issues outlined above.} 

These efforts take place at two complimentary frontiers. At the vanguard of the \emph{energy frontier} is the Large Hadron Collider, where protons are collided at energies never reached before in any experimental settings. The CMS and ATLAS experiments look for new, heavy particles produced in these unprecedented circumstances. However, so far it has only been possible to \emph{rule out} alternative theories, excluding regions of (infinite) phase space in a multitude of possible standard model extensions.\footnote{The, obviously extremely important, observation of the Higg's boson and the determination of its couplings and properties all agree with the Standard Model expectations.} 

The other frontier is the \emph{precision frontier}, which seeks to exploit that the (potential) existence of heavy particles can cause or influence phenomena at energies that are orders of magnitude smaller than required for their direct production. The canonical example is that of weak decays that occur in atom nuclei at rest, but are mediated by the $W$ boson, which can only be directly produced in powerful particle accelerators. Thus, by way of precise measurements of processes at a low energy, characteristics of high energy physics can be derived. The field of flavour physics, which concerns itself with processes that distinguish the different generations of quarks and leptons in the Standard Model, plays a significant role at the precision frontier. Historically, both the existence of the charm and third-generation quarks were predicted before the particles could be produced, in order to explain lower energy phenomena (the lack of flavour-changing neutral currents, and \CP violation, respectively); and the $c$ and $t$ masses could be constrained before their discoveries by mixing measurements of $\Delta m_K$ and $\Delta m_B$, respectively. With regards to constraints on \emph{new}, as-of-yet unknown physics effects, impressive results have been achieved in flavour physics experiments, where results of meson mixing and \CP violation measurements provide bounds on BSM physics at energy scales of the order $\Lambda \gtrsim 10^{4}\tev$; a much higher energy scale that what can be directly probed in current and potential particle colliders.\footnote{These bounds do not rule out new physics lower energy scales, but they do impose stringent constraints on the possible flavour structure of any new physics model at the $<10^{4}\tev$ scale.}

This thesis places itself at the forefront of the latter efforts, presenting the World's most precise measurement of the \CP-violating phase $\gamma$; a fundamental parameter in the Standard Model, in which it describes the sole source of matter-antimatter asymmetry. The measurement is based on samples of \BtoDK and \BtoDpi decays, where the \D meson is reconstructed in one of the final states \Kspipi and \KsKK. The role of \CP violation in the Standard Model and the methodology uses to probe it in the thesis are described in detail in Chapter~\ref{ch:2-litreview}.

The measurement is based on data taken with the \lhcb experiment during Run~1~and~2 of the LHC. The detector and software used to obtain the data set are described in Chapter~\ref{ch:3-detector}.

It is the first time that \BtoDpi decays are used to measure $\gamma$ with the approach taken in the thesis; therefore, a number of effects had to be considered that were not important in measurements based solely on \BtoDK decays. The most important such effect is due to \CP violation and material interaction of the neutral kaon in the final state, because existing literature at the outset of the work suggested that the impact could potentially be significant. Therefore, these phenomena had to be analysed in detail to establish the feasibility of the main measurement of the thesis. Chapter~\ref{ch:4-KS-CPV} describes such an analysis, and shows that the existing literature overestimated the potential effect by an order of magnitude; in fact the impact on the measurement is negligible.

The main results of the thesis are presented in Chapter~\ref{ch:5-GGSZ-measurement}, where the measurement of $\gamma$ is described in detail. The approach differs from earlier measurements in the same decay channel due to the inclusion of \BtoDpi as a signal channel; therefore, a series of feasibility studies that lead to the specific approach being chosen are also presented.

%Do mention the Belle~\cite{BelleCombo} and BaBar~\cite{BabarCombo} combinations of $\gamma$ measurements, including which decay channels they include (maybe

Naturally, the work presented here is far from the last word to be said on the value of $\gamma$. In the very near future, several important results will be updated by the \lhcb collaboration, based on the full data set collected during Run~2 of the LHC. In the coming 10--15 years, both the \lhcb and Belle~II collaborations will record data samples of \B decays that are orders of magnitudes larger than those collected before, pushing the obtainable precision on $\gamma$ towards, even below, one degree. An outlook towards this ultra-high-precision era of \CP-violation measurements is given in Chapter~\ref{ch:6-conclusion}, along with a summary of the contributions made in the thesis.



% subsection structure_of_the_thesis (end)