This thesis presents a measurement of the Cabibbo-Kobayashi-Maskawa angle $\gamma$ using \BtoDK and \BtoDpi decays, where the \D meson decays to one of the final states \Kspipi and \KsKK.  The measurement relies on the distribution of signal decays over the phase space of the \D decay, analysed using a model-independent method based on strong-phase measurements by the CLEO and BESIII collaborations. The measurement is performed using proton-proton collision data collected by the \lhcb experiment during the full Run 1~and~2 of the Large Hadron Collider, corresponding to a total integrated luminosity of 8.7\invfb at centre-of-mass energies of $\sqrt s =7$, $8$, and $13\tev$. The measurement determines that $\gamma= (68.7^{+5.2}_{-5.1})^\circ$, with an alternative solution corresponding to $\gamma+180^\circ$. This is the most precise stand-alone measurement of $\gamma$ to date, and achieves a precision that is comparable to all earlier measurements of $\gamma$ combined. 

The thesis also presents a phenomenological study of the impact of neutral kaon \CP violation and material interaction on $\gamma$ measurements with ${\Bpm\to (\KS h^+h^-)_D h'^\pm}$ decays. The existing literature at the outset of the thesis work had estimated the potential bias to be $\mathcal O(1^\circ)$ in \BtoDK decays and to scale with $1/r_B$. This suggests potentially large biases for a measurement with \BtoDpi decays, since $r_B^{D\pi}\simeq0.005$ is much smaller than $r_B^{DK}\simeq 0.1$. However, the thesis argues that the actual impact is an order of magnitude smaller, as long as the \CP-violation observables are determined based on the phase-space distribution of signal decays. This is demonstrated in a number of numerical studies that take the geometries of the \lhcb and Belle~II detectors into account.