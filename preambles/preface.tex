The work presented in this thesis has been resulted in two papers, either under review or published in the Journal of High Energy Physics. These are
\begin{itemize}

    \item [] \cite{GGSZ-B2Dh} \emph{Measurement of the CKM angle $\gamma$ using $\Bpm\to [\KS h^+h^-]_\D h^\pm$ decays}, submitted to JHEP. \\
    This paper describes a measurement of the CKM angle $\gamma$ using $pp$ collision data taken with the LHCb experiment during the Run~1 of the LHC, in 2011 and 2012, and during the full Run~2, in 2015--2018. The measurement uses the decay channels $\Bpm\to\D h^\pm$ where $\D\to\KS h'^+h'^-$, in which $h$ and $h'$ denotes pions or kaons. It obtains a value of $\gamma = (69\pm5)^\circ$, which constitutes the world's best single-measurement determination of $\gamma$. The work is the main focus of this thesis and described in detail in Chapter~\ref{ch:5-GGSZ-measurement}.

    \item [] \cite{KsCPV} \emph{CP violation and material interaction of neutral kaons
                        in measurements of the CKM angle $\gamma$ using $B^\pm\to
                        DK^\pm$ decays where $D\to K_\text{S}^0\pi^+\pi^-$}, JHEP 19 (2020) 106. \\
                        This paper describes a phenomenological study of the impact of neutral-kaon \CP violation and material interaction on measurements of $\gamma$. With the increased measurement precision to come in the near future, an understanding of these effects is crucial, especially in the context of $\B\to\D\pi$ decays; however no detailed study had been published at the start of this thesis. The study is the subject of Chapter~\ref{ch:4-KS-CPV}. Some text excerpts and figures from the paper have been reproduced in the thesis.
\end{itemize}
All of the work described in the thesis is my own, except where clearly referenced to others. Furthermore, I contributed significantly to an analysis of $\Bpm\to\D\Kpm$ decays with LHCb data taken in 2015 and 2016, now published in
\begin{itemize}
    \item [] \cite{LHCb-PAPER-2018-017} \emph{Measurement of the CKM angle $\gamma$ using \decay{\Bpm}{DK^\pm} with \decay{D}{\KS \pi^+ \pi^-}, $\KS K^+K^-$ decays}, JHEP 08 (2018) 176.
\end{itemize}
I was responsible for the selection and analysis of the signal channel, studies of systematic uncertainties, and the interpretation of the measured observables in terms of underlying physics parameters. The measurement is superseded by that of Ref.~\cite{GGSZ-B2Dh} and is not described in detail in the thesis.

Within the \lhcb collaboration, I took part in the preparation for Run~3 by working as a \emph{migration coordinator} for the \emph{\B-decay-to-open-charm} (B2OC) physics working group, responsible\footnote{Along with Alessandro Bertolin and Shunan Shang.} for the migration of the working group's centralised, offline selections (so called \emph{stripping lines}) to the software-trigger framework being developed for the \lhcb Upgrade.\footnote{The current software trigger and \emph{stripping} framework is presented in Section~\ref{sec:the_lhcb_triggerring_system}.} With more than 800 lines in the current B2OC stripping module this is a major task; I took a leading role in the initial design and testing of the upgraded B2OC module, and in helping the first analysts implement their selections within it. I have also undertaken shift work as RICH piquet and Data Manager, and acted as the liason between the B2OC physics working group and the \emph{particle-identification} performance working group.

Beyond the data analysis and phenomenology work that I have performed I have made numerous other contributions to the LHCb experiment. I have undertaken shift work as a RICH piquet and also as Data Manager. For two years i was the liaison between the particle-identification performance working group and the B2OC physics working group. This required ensuring the news and updates were communicated. The exchange of this information is critical since the PID calibration is updated regularly during data-taking and can also be reprocessed to correct for mistakes and issue found. In this role i also performed some validation work. The PID performance working group is also where liaisons from different working groups come together and therefore I was able to showcase ideas or problems found in other working groups for the benefit of analysts in the B2OC working group. 

Preparations for Run 3 are well underway. A major part of the upgrade is the evolution of the trigger system. The entire trigger moves to software only and the whole detector is read out upfront in order to make the 1st level of selections. With an average of 7 pp interactions per bunch crossing an event is very likely to have either a bb-bar or cc-bar pair produced, and hence the role of trigger becomes one where the interesting signal must be separated from other signal rather than the one where signal is separated from background. The computing model is limited to xx MB/s to storage and so it is critical that that limited bandwidth is well utilised. Secondly the computations at the second layer of software trigger will perform the full reconstruction but still need to be fast enough to process the data before the disk buffer is exceeded. 

During Run2 the main concept was inclusive triggering, with the stripping run subsequently to select out the signals of interest. In Run3 an equivalent of this stage must run in the second stage of the software trigger, and in order to meet storage requirements keep the signal candidate and track only rather than full event information. Amongst the working groups B2OC is unique in have a large single set of code which is able to be configured to select a large number of decay channels (approx 800 in total), in the stripping. For Run3 the equivalent action must take place at the second stage of the HLT2 trigger, and hence is subject to an enlarged set of constraints in terms of timing and output. I took the role of migration coordinator to be responsible for the migration of the working groups centralised offline selection to the software trigger framework, being developed for LHCb Upgrade. With the scope of the physics of the 800 lines this was a major task in the conceptual design of the model. In particular the advantage of the B2OC set of lines is their similarity and to try and exploit the making of combinatorics to run as fast as possible. For example D meson candidates are required to be built only once and then used by multiple B decay. The design has to be flexible enough to allow for user choice but remain inclusive to keep the overall time low. I had a leading role in the initial design and testing of the B2OC module and helped the first analysts implement the final selections within it.
