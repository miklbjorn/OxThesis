The work presented in this thesis has resulted in two papers, either to be submitted to or published in the Journal of High Energy Physics. These are
\begin{itemize}

    \item [] \cite{GGSZ-B2Dh} \emph{Measurement of the CKM angle $\gamma$ using \BtoDK and \BtoDpi with $\D\to\KS h^+ h^-$ decays}, to be submitted to JHEP. \\
    This paper describes a measurement of the CKM angle $\gamma$ using $pp$ collision data taken with the LHCb experiment during the Run~1 of the LHC, in 2011 and 2012, and during the full Run~2, in 2015--2018. The measurement uses the decay channels $\Bpm\to\D h^\pm$ where $\D\to\KS h'^+h'^-$, in which $h$ and $h'$ denotes pions or kaons. It obtains a value of $\gamma= (68.7^{+5.2}_{-5.1})^\circ$, which constitutes the World's best single-measurement determination of $\gamma$. The work is the main focus of this thesis and described in detail in Chapter~\ref{ch:5-GGSZ-measurement}.

    \item [] \cite{KsCPV} \emph{CP violation and material interaction of neutral kaons
                        in measurements of the CKM angle $\gamma$ using $B^\pm\to
                        DK^\pm$ decays where $D\to K_\text{S}^0\pi^+\pi^-$}, JHEP 19 (2020) 106. \\
                        This paper describes a phenomenological study of the impact of neutral-kaon \CP violation and material interaction on measurements of $\gamma$. With the increased measurement precision to come in the near future, an understanding of these effects is crucial, especially in the context of $\B\to\D\pi$ decays; however no detailed study had been published at the start of this thesis. The study is the subject of Chapter~\ref{ch:4-KS-CPV}. Some text excerpts and figures from the paper have been reproduced in the thesis.
\end{itemize}
All of the work described in the thesis is my own, except where clearly referenced to others. Furthermore, I contributed significantly to an analysis of $\Bpm\to\D\Kpm$ decays with LHCb data taken in 2015 and 2016, now published in
\begin{itemize}
    \item [] \cite{LHCb-PAPER-2018-017} \emph{Measurement of the CKM angle $\gamma$ using \decay{\Bpm}{DK^\pm} with \decay{D}{\KS \pi^+ \pi^-}, $\KS K^+K^-$ decays}, JHEP 08 (2018) 176.
\end{itemize}
I was responsible for the selection and analysis of the signal channel, studies of systematic uncertainties, and the interpretation of the measured observables in terms of underlying physics parameters. The measurement is superseded by that of Ref.~\cite{GGSZ-B2Dh} and is not described in detail in the thesis.

Beyond my data analysis and phenomenology work, I have made numerous other contributions to the LHCb experiment. I took part in the preparation for Run~3 by working as a \emph{migration coordinator} for the \emph{\B-decay-to-open-charm} (B2OC) physics working group, responsible\footnote{Along with Alessandro Bertolin and Shunan Shang.} for the development of the working group's centralised selections in the software trigger framework being developed for the \lhcb Upgrade.\footnote{The current software trigger and centralised selection framework is presented in Section~\ref{sec:the_lhcb_triggerring_system}.} With more than 800 lines in the current B2OC selection module this is a major task. The new B2OC selection framework was redesigned and written from scratch, and I took a leading role in the initial design and testing, and in helping the first analysts implement their selections within it. I have also undertaken shift work as RICH piquet and Data Manager, and acted as the liason between the B2OC physics working group and the \emph{particle-identification} performance working group. This work is described in more detail in Appendix~\ref{cha:contribution_for_the_lhcb_collaboration}.


